\begin{titlepage}
  \begin{center}

  {\Huge FAST\_AXIS\_UART}

  \vspace{25mm}

  \includegraphics[width=0.90\textwidth,height=\textheight,keepaspectratio]{img/AFRL.png}

  \vspace{25mm}

  \today

  \vspace{15mm}

  {\Large Jay Convertino}

  \end{center}
\end{titlepage}

\tableofcontents

\newpage

\section{Usage}

\subsection{Introduction}

\par
Simple FAST UART core for TTL rs232 software mode data communications. No hardware handshake.
This contains its own internal baud rate generator that creates an enable to allow data output
or sampling. Baud clock and aclk can be the same clock.

\subsection{Dependencies}

\par
The following are the dependencies of the cores.

\begin{itemize}
  \item fusesoc 2.X
  \item iverilog (simulation)
  \item cocotb (simulation)
\end{itemize}

\subsubsection{fusesoc\_info Depenecies}
\begin{itemize}
\item dep
	\begin{itemize}
	\item AFRL:clock:mod\_clock\_ena\_gen:1.1.1
	\item AFRL:utility:helper:1.0.0
	\item AFRL:simple:piso:1.0.0
	\item AFRL:simple:sipo:1.0.0
	\end{itemize}
\end{itemize}


\subsection{In a Project}
\par
This core connects a UART to the AXIS bus. Meaning this is a streaming device only. Connect the RX/TX to the UART in question and connect the AXIS to its intended endpoints.

\section{Architecture}
\par
This core is made up of other cores that are documented in detail in there source. The cores this is made up of are the,
\begin{itemize}
  \item \textbf{fast\_axis\_uart} Interface with UART and present the data over AXIS interface (see core for documentation).
  \item \textbf{mod\_clk\_gen\_ena} Generates enable pulses at the baud rate based on the input clock.
  \item \textbf{PISO} Take parallel input data and output in a serial fashion.
  \item \textbf{SIPO} Take serial data input and output parallel data.
\end{itemize}

\section{Building}

\par
The FAST AXIS UART is written in Verilog 2001. It should synthesize in any modern FPGA software. The core comes as a fusesoc packaged core and can be included in any other core. Be sure to make sure you have meet the dependencies listed in the previous section. Linting is performed by verible using the lint target.

\subsection{fusesoc}
\par
Fusesoc is a system for building FPGA software without relying on the internal project management of the tool. Avoiding vendor lock in to Vivado or Quartus.
These cores, when included in a project, can be easily integrated and targets created based upon the end developer needs. The core by itself is not a part of
a system and should be integrated into a fusesoc based system. Simulations are setup to use fusesoc and are a part of its targets.

\subsection{Source Files}

\subsubsection{fusesoc\_info File List}
\begin{itemize}
\item src
	\begin{itemize}
	\item src/fast\_axis\_uart.v
	\end{itemize}
\item tb\_cocotb\_full
	\begin{itemize}
	\item {'tb/tb\_cocotb\_full.py': {'file\_type': 'user', 'copyto': '.'}}
	\item {'tb/tb\_cocotb\_full.v': {'file\_type': 'verilogSource'}}
	\end{itemize}
\end{itemize}


\subsection{Targets}

\subsubsection{fusesoc\_info Targets}
\begin{itemize}
\item default
	\begin{itemize}
	\item[$\space$] Info: Default for IP intergration.
	\end{itemize}
\item lint
	\begin{itemize}
	\item[$\space$] Info: Lint with Verible
	\end{itemize}
\item sim\_cocotb\_full
	\begin{itemize}
	\item[$\space$] Info: Cocotb unit tests
	\end{itemize}
\end{itemize}


\subsection{Directory Guide}

\par
Below highlights important folders from the root of the directory.

\begin{enumerate}
  \item \textbf{docs} Contains all documentation related to this project.
    \begin{itemize}
      \item \textbf{manual} Contains user manual and github page that are generated from the latex sources.
    \end{itemize}
  \item \textbf{src} Contains source files for the core
  \item \textbf{tb} Contains test bench files for iverilog and cocotb
    \begin{itemize}
      \item \textbf{cocotb} testbench files
    \end{itemize}
\end{enumerate}

\newpage

\section{Simulation}
\par
There are a few different simulations that can be run for this core.

\subsection{cocotb}
\par
To use the cocotb tests you must install the following python libraries.
\begin{lstlisting}[language=bash]
  $ pip install cocotb
  $ pip install cocotbext-axi
\end{lstlisting}

Each module has a cocotb based simulation. These use the cocotb extensions made by Alex.
The two extensions used are cocotbext-axi and cocotbext-uart. These provide outside verification
of the implimentation. These tests consist of 3 different fusesoc targets.

\begin{itemize}
  \item \textbf{sim\_cocotb\_full} Standard simulation of TX/RX passing data to and from cocotbexts.
\end{itemize}

Then you must use the cocotb sim target. The targets above can be run with various bus and fifo parameters.
\begin{lstlisting}[language=bash]
  $ fusesoc run --target AFRL:device_converter:fast_axis_uart:1.0.0
\end{lstlisting}

\newpage

\section{Module Documentation} \label{Module Documentation}

\begin{itemize}
\item \textbf{fast\_axis\_uart} Wrapper for all of the modules to create a singular device to interface with.\\
\end{itemize}


